\documentclass{scrartcl} 

\usepackage[english]{babel} 

\newcommand{\fline}{\mathcal{L}} % finite line 

%% General packages
\usepackage[utf8]{inputenc} % For coding and special characters
\usepackage[T1]{fontenc}
\usepackage{lmodern} % Fancier font
\usepackage[top=3cm,bottom=3cm,left=2.5cm,right=2.5cm]{geometry} 

\usepackage{graphicx} 
\usepackage{blindtext} 

%% Colors
\usepackage{xcolor} % Farben
\definecolor{aaudarkblue}{RGB}{67, 111, 128}
\definecolor{aaublue}{RGB}{83, 159, 198}
\definecolor{aaulightgray}{RGB}{232,232,232}
\definecolor{aaugray}{RGB}{98,98,98}

%% Design Choices
\usepackage[labelfont=bf]{caption}
\usepackage{enumitem}
\usepackage{float}
\setlist[itemize]{label=$\blacktriangleright$}

%% Header and footer
\usepackage[autooneside=false, automark]{scrlayer-scrpage} 
\pagestyle{scrheadings} % Activate style
\clearpairofpagestyles % Clear current setting
\automark[section]{section} % set \headmark to contain the headings of \section
\ihead{\textnormal{\headmark}} % reference to the adjusted headmark at the inside of the head
\ohead[\pagemark]{\pagemark} % reference to the page number at the outside of the head
\KOMAoptions{headsepline = true} % Activate seperation line for the header

%% (Framed) theorem environments
\usepackage[framemethod=TikZ]{mdframed}
\mdfdefinestyle{theoremstyle}{ % Define new style for theorems
	linecolor=aaublue,
	linewidth=1pt,
	frametitlerule=false,%
	apptotikzsetting={\tikzset{mdfframetitlebackground/.append style={
	color=aaublue!65}}},
	innertopmargin=\topskip,
	innerbottommargin=\topskip,
	nobreak=false,
	backgroundcolor=aaublue!15,
	frametitlefont=\bfseries,
	theoremtitlefont=\normalfont,
	theoremseparator = {},
	footnoteinside=false
}

% Theorem
\mdtheorem[style=theoremstyle]{theoremx}[examplex]{Theorem} 
\newenvironment{theorem}[1][]{% Add brackets around caption for the numbered environment
    \begin{theoremx}[\ifx&#1&\else(#1)\fi]%
}{%
    \end{theoremx}%
}
\newenvironment{theorem*}[1][]{% add brackets around caption for the unnumbered environment
    \begin{theoremx*}[\ifx&#1&\else(#1)\fi]%
}{%
    \end{theoremx*}%
}

% Propositon
\mdtheorem[style=theoremstyle]{propositionx}[examplex]{Proposition} 
\newenvironment{proposition}[1][]{% Add brackets around caption for the numbered environment
    \begin{propositionx}[\ifx&#1&\else(#1)\fi]%
}{%
    \end{propositionx}%
}
\newenvironment{proposition*}[1][]{% add brackets around caption for the unnumbered environment
    \begin{propositionx*}[\ifx&#1&\else(#1)\fi]%
}{%
    \end{propositionx*}%
}

% Lemma
\mdtheorem[style=theoremstyle]{lemmax}[examplex]{Lemma} 
\newenvironment{lemma}[1][]{% Add brackets around caption for the numbered environment
    \begin{lemmax}[\ifx&#1&\else(#1)\fi]%
}{%
    \end{lemmax}%
}
\newenvironment{lemma*}[1][]{% add brackets around caption for the unnumbered environment
    \begin{lemmax*}[\ifx&#1&\else(#1)\fi]%
}{%
    \end{lemmax*}%
}

% Proposition
\mdtheorem[style=theoremstyle]{corollaryx}[examplex]{Corollary} 
\newenvironment{corollary}[1][]{% Add brackets around caption for the numbered environment
    \begin{corollaryx}[\ifx&#1&\else(#1)\fi]%
}{%
    \end{corollaryx}%
}
\newenvironment{corollary*}[1][]{% add brackets around caption for the unnumbered environment
    \begin{corollaryx*}[\ifx&#1&\else(#1)\fi]%
}{%
    \end{corollaryx*}%
}

\mdfdefinestyle{definitionstyle}{ % Define new style for definitions
	linecolor=aaulightgray!135,
	linewidth=1pt,
	frametitlerule=false,%
	apptotikzsetting={\tikzset{mdfframetitlebackground/.append style={
	color=aaulightgray}}},
	innertopmargin=\topskip,
	innerbottommargin=\topskip,
	nobreak=false,
	backgroundcolor=aaulightgray!50,
	frametitlefont=\bfseries,
	theoremtitlefont=\normalfont,
	theoremseparator = {},
	footnoteinside=false
}

% Definition
\mdtheorem[style=definitionstyle]{definitionx}[examplex]{Definition}
\newenvironment{definition}[1][]{% Add brackets around caption for the numbered environment
    \begin{definitionx}[\ifx&#1&\else(#1)\fi]%
}{%
    \end{definitionx}%
}
\newenvironment{definition*}[1][]{% add brackets around caption for the unnumbered environment
    \begin{definitionx*}[\ifx&#1&\else(#1)\fi]%
}{%
    \end{definitionx*}%
}

% Example
\newtheoremstyle{examplestyle} % Define new style for examples (bold heading, linebreak, normal text)
	{\topsep}{\topsep}
	{}{}
	{\bfseries}{}
	{\newline}{}
\theoremstyle{examplestyle} % Activate style
\newtheorem{examplex}{Example} % New environment for numbered examples
\newtheorem*{examplexx}{Example} % New environment for unnumbered examples
\newenvironment{example} % Add triangle at the end of the numbered example environment
	{\pushQED{\qed}\renewcommand{\qedsymbol}{$\triangle$}\examplex}
	{\popQED\endexamplex}
\newenvironment{example*} % Add triangle at the end of the unnumbered example environment
	{\pushQED{\qed}\renewcommand{\qedsymbol}{$\triangle$}\examplexx}
	{\popQED\endexamplexx}
  
% Remark/Notation
\newtheoremstyle{remarkstyle} % Define new style for remarks (bold heading, nomral text)
	{\topsep}{\topsep}
	{}{}
	{\bfseries}{.}
	{5pt plus 1pt minus 1pt}{}
\theoremstyle{remarkstyle} % Activate style
\newtheorem{remarkx}[examplex]{Remark} % New environment for numbered remarks
\newtheorem*{remarkxx}{Remark} % New environment for unnumbered remarks
\newenvironment{remark} % Add diamond at the end of an numbered remark environment
	{\pushQED{\qed}\renewcommand{\qedsymbol}{$\diamond$}\remarkx}
	{\popQED\endremarkx}
\newenvironment{remark*} % Add diamond at the end of an unnumbered remark environment
	{\pushQED{\qed}\renewcommand{\qedsymbol}{$\diamond$}\remarkxx}
	{\popQED\endremarkxx}

\newtheorem{notation}[examplex]{Notation} % New environments for numbered notation
\newtheorem*{notation*}{Notation} % New environments for unnumbered notation

%% References
\usepackage[
	backend=biber, 
	bibstyle=numeric,
	natbib=true, 
	hyperref=true,
]{biblatex} 
\let\oldprintbibliography\printbibliography
\renewcommand{\printbibliography}{% adds new page and correct reference for the table of contents
	\newpage \phantomsection
	\addcontentsline{toc}{section}{References}
	\oldprintbibliography
}
\usepackage{csquotes}

\usepackage{hyperref}  

\title{This are my lecture notes}
\subtitle{Title of the lecture, ST 2024}
\titlehead{\hfill\includegraphics[height=2cm]{fig/aau-logo.pdf}}
\author{Markus Tripp\thanks{Alpen-Adria-Universität Klagenfurt, 9020 Klagenfurt, Austria,~\href{mailto:markus.tripp@aau.at}{\texttt{markus.tripp@aau.at}.}}}

\begin{document}

\begin{titlepage} 
	\maketitle
	
	\begin{figure}[ht]
\centering
\scalebox{0.35}{
\begin{tikzpicture}
		% Background boxes
		\filldraw [line width=1pt, color=aaublue, fill=aaublue!15, rounded corners] (-6,-6)  rectangle (14,6);
		\filldraw [line width=1pt, color=aaudarkblue, fill=aaudarkblue!15, rounded corners] (6,-3)  rectangle (13,5);
		\filldraw [line width=1pt, color=aaulightgray!135, fill=aaulightgray!50, rounded corners] (-5,-5)  rectangle (5,5);
		
		\node[above, aaublue] at(-3,5) {$\dots$ Kakeya problem};
		
		% EUCLIDIAN
		% Creating the axis and the grid
        	\draw[aaublue,very thin] (-4,-4) grid (4,4);
        	\foreach \i in {-3,...,3} \draw (\i,-.1)--(\i,.1);
        	\foreach \i in {-3,...,3} \draw (-.1,\i)--(.1,\i);
        	\node[below] at (3.75,0) {$1$};
       	\node[right] at (0,3.75) {$1$}; 
        	\draw[->] (-4,0) -- (4,0);
        	\draw[->] (0,-4) -- (0,4);
        
        	% Creating the deltoid curve, circle and the unit line segment
        	\draw (0,0) circle (3);
        	\def\a{1} \def\b{3}
        	\draw[line width=1.5pt, aaudarkblue] plot[samples=100, domain=0:360, smooth, variable=\t] ({(\b-\a)*cos(\t)+\a*cos((\b-\a)*\t/\a},{(\b-\a)*sin(\t)-\a*sin((\b-\a)*\t/\a});
        	\draw[line width=1.5pt] (-1,0) -- (3,0);
        	\node[above, font=\boldmath] at (0.5,0) {$1$};
        
        	% Label
        	\node[above] at (-3,4) {Euclidean};
        
        	% FINITE
        	\foreach \i in {8, 9, 10, 12} \filldraw (\i,3) circle(0.1); % first row
        	\foreach \i in {11} \draw (\i,3) circle(0.1);
        	\foreach \i in {9, 11} \filldraw (\i,2) circle(0.1); % second row
        	\foreach \i in {8, 10, 12} \draw (\i,2) circle(0.1);
        	\foreach \i in {9, 10, 12} \filldraw (\i,1) circle(0.1); % third row
        	\foreach \i in {8, 11} \draw (\i,1) circle(0.1);
        	\foreach \i in {9, 11, 12} \filldraw (\i,0) circle(0.1); % fourth row
        	\foreach \i in {8, 10} \draw (\i,0) circle(0.1);
        	\foreach \i in {8,...,12} \filldraw (\i,-1) circle(0.1); % fifth row
        
        	\draw (7.5,3.5) -- (7.5,-2.5);
        	\draw (6.5,-1.5) -- (12.5,-1.5);
        
        	% Label
        	\node at (7,3) {$4$};
        	\node at (7,2) {$3$};
        	\node at (7,1) {$2$};
        	\node at (7,0) {$1$};
        	\node at (7,-1) {$0$};
        
        	\node at (8,-2) {$0$};
        	\node at (9,-2) {$1$};
        	\node at (10,-2) {$2$};
        	\node at (11,-2) {$3$};
        	\node at (12,-2) {$4$};
        
        	\node[above, aaudarkblue] at (7.25,4) {Finite};
\end{tikzpicture}
}
\end{figure}

	\thispagestyle{empty}
\end{titlepage}

\tableofcontents 

\newpage
\section{Preliminaries}
\label{sec:prelim}
All sorts of things are explained and introduced here. One can also put stuff in fancy boxes. A tasty example:

\begin{definition}[Group action]
	Let $G$ be a group and $X$ a set. A mapping $\alpha:G\times X \to X$ that satisfies
	\begin{enumerate}[label=(A\arabic*)]
		\item $\alpha(e,x) = x$,
		\item $\alpha(g,\alpha(h,x))=\alpha(gh,x)$
	\end{enumerate}
	for all $g$, $h \in G$ and $x \in X$ is called a \emph{group action} of $G$ on $X$.
\end{definition}

\begin{remark}
	It is very common that one replaces $\alpha$ with a dot. Then the two above axioms read as $e\cdot x = x$ and $g\cdot (h \cdot x) = (gh) \cdot x$.
\end{remark}

\section{Digression on finite Kakeya sets}
\label{sec:kakeya}

\begin{example}[Kakeya set in $\F_3^2$]\label{ex:kakeya-set}
	We consider the finite field $\F_3 = \Z/3\Z = \{ 0+3\Z,\, 1+3\Z,\, 2+3\Z \}$. 
	For the sake of readability, we will denote the representative for each residue class in the following discussion. 
	Our goal is to find a non-trivial Kakeya set, denoted as $K \subseteq \F_3^2$, which contains a line in each direction. 
	The directions we are considering are as follows:  $\left[ (1,0) \right] = \{ (1,0),\, (2,0) \}$, 
	$\left[ (0,1) \right] = \{ (0,1),\, (0,2) \}$, $\left[ (1,1) \right] = \{ (1,1),\, (2,2) \}$, and 
	$\left[ (1,-1) \right] = \{ (1,2),\, (2,1) \}$ (note: $(1,-1) = (1,2)$). 
	As an example, we select the following lines:
	\begin{align*}
		\fline_{(1,0),\, (0,0)} &= \{ (0,0) + t(1,0) : t \in \{0,\, 1, \, 2\} \} = \{ (0,0),\, (1,0),\, (2,0) \}, \\
		\fline_{(0,1),\, (2,0)} &= \{ (2,0),\, (2,1),\, (2,2) \}, \quad \fline_{(1,1),\, (0,0)} = \{ (0,0),\, (1,1),\, (2,2) \} \\
		\fline_{(1,-1),\, (1,2)} &= \{ (1,2),\, (2,1),\, (0,0) \},
	\end{align*}
	resulting in the Kakeya set $K \subseteq \F_3^2$:
	\begin{align*}
		K &\coloneqq \fline_{(1,0),\, (0,0)} \cup \fline_{(0,1),\, (2,0)} \cup \fline_{(1,1),\, (0,0)} \cup \fline_{(1,-1),\, (1,2)} \\
		&= \{ (0,0),\, (1,0),\, (1,1),\, (1,2),\, (2,0),\, (2,1),\, (2,2) \}.
	\end{align*}
\end{example}

\begin{figure}[ht]
\centering
\scalebox{0.75}{
\begin{tikzpicture}
	% Background box
	\filldraw [line width=1pt, color=aaublue, fill=aaublue!15, rounded corners] (-2,-8)  rectangle (18,3);
		
	% First grid
	\foreach \i in {0, 1, 2} \foreach \j in {0,1,2} \draw (\i,\j) circle(0.1);
	\foreach \i in {0, 1, 2} \node at (\i,-1) {$\i$};
	\foreach \j in {0, 1, 2} \node at (-1,\j) {$\j$};
	\draw (-1.5,-0.5) -- (2.5,-0.5);
	\draw (-0.5,-1.5) -- (-0.5,2.5);
	\node at (3,1) {$\rightsquigarrow$};
	% Kakeya property
	\foreach \i in {0,1,2} \filldraw (\i,0) circle(0.1);
	\draw[dashed] (-0.25,0) -- (2.25,0);
	\node at (0.5,-2) {$v=(1,0)$, $w=(0,0)$};
        
	% Second grid
	\foreach \i in {5, 6, 7} \foreach \j in {0,1,2} \draw (\i,\j) circle(0.1);
	\foreach \i[evaluate={\k=int(\i -5);}] in {5,6,7} \node at (\i,-1) {$\k$};
	\foreach \j in {0,1,2} \node at (4,\j) {$\j$};
	\draw (3.5,-0.5) -- (7.5,-0.5);
	\draw (4.5,-1.5) -- (4.5,2.5);
	\node at (8,1) {$\rightsquigarrow$};
	% Kakeya property
	\foreach \j in {0,1,2} \filldraw (7,\j) circle(0.1);
	\draw[dashed] (7,-0.25) -- (7,2.25);
	\node at (5.5,-2) {$v=(0,1)$, $w=(2,0)$};
        
	% Third grid
	\foreach \i in {10, 11, 12} \foreach \j in {0, 1, 2} \draw (\i,\j) circle(0.1);
	\foreach \i[evaluate={\k=int(\i -10);}] in {10, 11, 12} \node at (\i,-1) {$\k$};
	\foreach \j in {0,1,2} \node at (9,\j) {$\j$};
	\draw (8.5,-0.5) -- (12.5,-0.5);
	\draw (9.5,-1.5) -- (9.5,2.5);
	\node at (13,1) {$\rightsquigarrow$};
	% Kakeya property
	\foreach \i[evaluate={\j=int(\i -10);}] in {10, 11, 12} \filldraw (\i,\j) circle(0.1);
	\draw[dashed] (9.75,-0.25) -- (12.25,2.25);
	\node at (10.5,-2) {$v=(1,1)$, $w=(0,0)$};
        
	% Fourth grid
	\foreach \i in {15, 16, 17} \foreach \j in {0, 1, 2} \draw (\i,\j) circle(0.1);
	\foreach \i[evaluate={\k=int(\i -15);}] in {15, 16, 17} \node at (\i,-1) {$\k$};
	\foreach \j in {0,1,2} \node at (14,\j) {$\j$};
	\draw (13.5,-0.5) -- (17.5,-0.5);
	\draw (14.5,-1.5) -- (14.5,2.5);
	% Kakeya property
	\filldraw (17,1) circle(0.1);
	\filldraw (16,2) circle(0.1);
	\filldraw (15,0) circle(0.1);
	\draw[dashed] (17.25,0.75) -- (15.75,2.25);
	\draw[dashed] (14.75,0.25) -- (15.25,-0.25);
	\node at (15.5,-2) {$v=(1,-1)$, $w=(1,2)$};
 		
	% Fifth grid
	\foreach \i in {7, 8, 9} \foreach \j in {-6, -5, -4} \draw (\i,\j) circle(0.1);
	\foreach \i[evaluate={\k=int(\i -7);}] in {7, 8, 9} \node at (\i,-7) {$\k$};
	\foreach \j[evaluate={\k=int(\j +6);}] in {-6, -5, -4} \node at (6,\j) {$\k$};
	\draw (5.5,-6.5) -- (9.5,-6.5);
	\draw (6.5,-7.5) -- (6.5,-3.5);
	\node at (5,-5) {$\rightsquigarrow$};
	% Kakeya property
	\foreach \i in {7,8,9} \filldraw (\i,-6) circle(0.1);
	\foreach \j in {-6,-5,-4} \filldraw (9,\j) circle(0.1);
	\foreach \i[evaluate={\j=int(\i -13);}] in {7,8,9} \filldraw (\i,\j) circle(0.1);
	\filldraw (9,-5) circle(0.1);
	\filldraw (8,-4) circle(0.1);
	\filldraw (7,-6) circle(0.1);
\end{tikzpicture}
}
\caption{Kakeya set in $\F_3^2$ of the Example~\ref{ex:kakeya-set}}
\label{fig:kakeya-set}
\end{figure}

\section{Improvements?}
\label{sec:improv}
Feel free to adapt/polish this template in any way you like. I am happy to discuss ideas and suggestions for general improvement of this template!

\printbibliography

\end{document}
