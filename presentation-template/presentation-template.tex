\documentclass{beamer}

\usepackage[english]{babel} 

\usetheme{aau}

\logo{
	\includegraphics[width=3cm]{fig/aau-logo.pdf}
}

\title[My presentation]{This is my presentation} 
\subtitle{A template for presentations}
\author[Markus Tripp]{Markus Tripp} 
\date[]{\today}
\titlegraphic{
	\begin{figure}[ht]
\centering
\scalebox{0.35}{
\begin{tikzpicture}
		% Background boxes
		\filldraw [line width=1pt, color=aaublue, fill=aaublue!15, rounded corners] (-6,-6)  rectangle (14,6);
		\filldraw [line width=1pt, color=aaudarkblue, fill=aaudarkblue!15, rounded corners] (6,-3)  rectangle (13,5);
		\filldraw [line width=1pt, color=aaulightgray!135, fill=aaulightgray!50, rounded corners] (-5,-5)  rectangle (5,5);
		
		\node[above, aaublue] at(-3,5) {$\dots$ Kakeya problem};
		
		% EUCLIDIAN
		% Creating the axis and the grid
        	\draw[aaublue,very thin] (-4,-4) grid (4,4);
        	\foreach \i in {-3,...,3} \draw (\i,-.1)--(\i,.1);
        	\foreach \i in {-3,...,3} \draw (-.1,\i)--(.1,\i);
        	\node[below] at (3.75,0) {$1$};
       	\node[right] at (0,3.75) {$1$}; 
        	\draw[->] (-4,0) -- (4,0);
        	\draw[->] (0,-4) -- (0,4);
        
        	% Creating the deltoid curve, circle and the unit line segment
        	\draw (0,0) circle (3);
        	\def\a{1} \def\b{3}
        	\draw[line width=1.5pt, aaudarkblue] plot[samples=100, domain=0:360, smooth, variable=\t] ({(\b-\a)*cos(\t)+\a*cos((\b-\a)*\t/\a},{(\b-\a)*sin(\t)-\a*sin((\b-\a)*\t/\a});
        	\draw[line width=1.5pt] (-1,0) -- (3,0);
        	\node[above, font=\boldmath] at (0.5,0) {$1$};
        
        	% Label
        	\node[above] at (-3,4) {Euclidean};
        
        	% FINITE
        	\foreach \i in {8, 9, 10, 12} \filldraw (\i,3) circle(0.1); % first row
        	\foreach \i in {11} \draw (\i,3) circle(0.1);
        	\foreach \i in {9, 11} \filldraw (\i,2) circle(0.1); % second row
        	\foreach \i in {8, 10, 12} \draw (\i,2) circle(0.1);
        	\foreach \i in {9, 10, 12} \filldraw (\i,1) circle(0.1); % third row
        	\foreach \i in {8, 11} \draw (\i,1) circle(0.1);
        	\foreach \i in {9, 11, 12} \filldraw (\i,0) circle(0.1); % fourth row
        	\foreach \i in {8, 10} \draw (\i,0) circle(0.1);
        	\foreach \i in {8,...,12} \filldraw (\i,-1) circle(0.1); % fifth row
        
        	\draw (7.5,3.5) -- (7.5,-2.5);
        	\draw (6.5,-1.5) -- (12.5,-1.5);
        
        	% Label
        	\node at (7,3) {$4$};
        	\node at (7,2) {$3$};
        	\node at (7,1) {$2$};
        	\node at (7,0) {$1$};
        	\node at (7,-1) {$0$};
        
        	\node at (8,-2) {$0$};
        	\node at (9,-2) {$1$};
        	\node at (10,-2) {$2$};
        	\node at (11,-2) {$3$};
        	\node at (12,-2) {$4$};
        
        	\node[above, aaudarkblue] at (7.25,4) {Finite};
\end{tikzpicture}
}
\end{figure}
}

\begin{document}

\maketitle

\section{Overview}
\begin{frame}{Overview}
\begin{enumerate}
	\item Notations and Definitions
	\item Results
	\begin{enumerate}[A]
		\item Orbit-Stabilizer Theorem
		\item Burnside Lemma
	\end{enumerate}
\end{enumerate}
\end{frame}

\section{Notations and Definitions}
\begin{frame}{A tasty slide!}
\begin{definition}[Group action]
	Let $G$ be a group and $X$ a set. A mapping $\alpha:G\times X \to X$ that satisfies
	\begin{cenumerate}{A}
		\item<2-> \tikz\node[initialtext] (A) {$\alpha(e,x) = x$};,
		\item<3-> $\alpha(g,\alpha(h,x))=\alpha(gh,x)$\tikz\node[initialtext] (B) {};
	\end{cenumerate}
	\onslide<4->{all $g$, $h \in G$ and $x \in X$ is called a \emph{group action} of $G$ on $X$.}
\end{definition} 
\onslide<5->{
	\tikz\node[targettext, right=10em of A, align=left] (C) {$e\cdot x=x$ \\ $g\cdot (h\cdot x) = (gh)\cdot x$};
	\begin{tikzpicture}[overlay, remember picture]
		\draw[textarrow] (A) to [bend left=15] node[midway, below] {reads as \dots} (C.north);
		\draw[textarrow] (B) to [bend right=15] node[pos=0.75, above left, yshift=-1ex] {reads as \dots} (C.south);
	\end{tikzpicture}
}	
\end{frame}

\begin{frame}{Improvements?}
\begin{block}{Improvements?}
	Feel free to adapt/polish this template in any way you like. I am happy to discuss ideas and suggestions for general improvement of this template!
\end{block}
\end{frame}

\end{document}