%% General packages
\usepackage[utf8]{inputenc} % For coding and special characters
\usepackage[T1]{fontenc} 
\usepackage{lmodern} % Fancier font
\usepackage[margin=0.9in, top=1.2in, bottom=1in]{geometry}

\usepackage{graphicx}  
\usepackage{blindtext}

\usepackage[textsize=tiny, textwidth=2cm]{todonotes}

%% Design choices
\usepackage[labelfont=bf]{caption}
\usepackage[most]{tcolorbox} 
\usepackage{enumitem}
\usepackage{float}
\setlist[itemize]{label=$\blacktriangleright$}
\usepackage{sectsty}
\sectionfont{\centering}

%% Colors
\usepackage{xcolor} 
\definecolor{logoblue}{RGB}{140,193,227}
\definecolor{logogray}{RGB}{232,232,232}
\definecolor{logopink}{RGB}{227, 140, 149}

%% (Framed) theorem environments
\usepackage[framemethod=TikZ]{mdframed}
\mdfdefinestyle{theoremstyle}{% Define new style for theorems
	linecolor=logoblue,
	linewidth=1pt,
	frametitlerule=false,%
	apptotikzsetting={\tikzset{mdfframetitlebackground/.append style={
	color=logoblue!65}}},
	innertopmargin=\topskip,
	innerbottommargin=\topskip,
	nobreak=false,
	backgroundcolor=logoblue!15,
	frametitlefont=\bfseries,
	theoremtitlefont=\normalfont,
	footnoteinside=false
}
\mdtheorem[style=theoremstyle]{theorem}[examplex]{Theorem} 
\mdtheorem[style=theoremstyle]{proposition}[examplex]{Proposition} 
\mdtheorem[style=theoremstyle]{lemma}[examplex]{Lemma} 
\mdtheorem[style=theoremstyle]{corollary}[examplex]{Corolllary} 

\mdfdefinestyle{definitionstyle}{% Define new style for definitions
	linecolor=logogray!135,
	linewidth=1pt,
	frametitlerule=false,
	apptotikzsetting={\tikzset{mdfframetitlebackground/.append style={
	color=logogray}}},
	innertopmargin=\topskip,
	innerbottommargin=\topskip,
	nobreak=false,
	backgroundcolor=logogray!50,
	frametitlefont=\bfseries,
	theoremtitlefont=\normalfont,
	footnoteinside=false
}
\mdtheorem[style=definitionstyle]{definition}[examplex]{Definition} 

\newtheoremstyle{examplestyle} % Define new style for examples (bold heading, linebreak, normal text)
	{\topsep}{\topsep}
	{}{}
	{\bfseries}{}
	{\newline}{}
\theoremstyle{examplestyle} % Activate style
\newtheorem{examplex}{Example} % New environment for examples
\newenvironment{example} % Add triangle at the end of an example environment
	{\pushQED{\qed}\renewcommand{\qedsymbol}{$\triangle$}\examplex}
	{\popQED\endexamplex}
  
\newtheoremstyle{remarkstyle} % Define new style for remarks (bold heading, nomral text)
	{\topsep}{\topsep}
	{}{}
	{\bfseries}{.}
	{5pt plus 1pt minus 1pt}{}
\theoremstyle{remarkstyle} % Activate style
\newtheorem{remarkx}[examplex]{Remark} % New environment for remarks
\newenvironment{remark} % Add diamond at the end of an remark environment
  {\pushQED{\qed}\renewcommand{\qedsymbol}{$\diamond$}\remarkx}
  {\popQED\endremarkx}
  
%% References
\usepackage[
	backend=biber, 
	bibstyle=numeric, 
	natbib=true, 
	hyperref=true, 
]{biblatex} 
\addbibresource{fancy-article-template.bib} % Include bib-file
\nocite{*}
\usepackage{csquotes}

\usepackage{hyperref} 

%% Make title
\newcommand{\myabstract}[1]{%
	\vspace{-6ex}
	\begin{tcolorbox}[colback = logopink!15, colframe = logopink, width=\linewidth, arc=1mm, auto outer arc]
		\normalsize\textsf{\textbf{Abstract.}}\small~#1
	\end{tcolorbox}
	\vspace{1.5ex}
}

\makeatletter
\newcommand{\blthanks}[1]{% Blind thanks (without marker and symbol)
	\protected@xdef\@thanks{\@thanks\protect\footnotetext[0]{#1}}}
\makeatother

\let\cdate\date % Save copy of \date
\renewcommand{\date}[1]{\cdate{\blthanks{\emph{Date:}~#1.}}} % Move date in footnote

\let\svthefootnote\thefootnote
\newcommand\freefootnote[1]{% Footnote without marker and number
  \let\thefootnote\relax%
  \footnotetext{#1}%
  \let\thefootnote\svthefootnote%
}
\newcommand{\keywords}[1]{\freefootnote{\emph{Key words and phrases:}~#1.}}
